\documentclass[a4paper,12pt,twoside]{memoir}

% Castellano
\usepackage[spanish,es-tabla]{babel}
\selectlanguage{spanish}
\usepackage[utf8]{inputenc}
\usepackage[T1]{fontenc}
\usepackage{lmodern} % Scalable font
\usepackage{microtype}
\usepackage{placeins}

\usepackage{listings}

\RequirePackage{booktabs}
\RequirePackage[table]{xcolor}
\RequirePackage{xtab}
\RequirePackage{multirow}

% Links
\usepackage[colorlinks]{hyperref}
\hypersetup{
	allcolors = {red}
}

% Ecuaciones
\usepackage{amsmath}

% Rutas de fichero / paquete
\newcommand{\ruta}[1]{{\sffamily #1}}

% Párrafos
\nonzeroparskip


% Imagenes
\usepackage{graphicx}
\newcommand{\imagen}[2]{
	\begin{figure}[!h]
		\centering
		\includegraphics[width=0.9\textwidth]{#1}
		\caption{#2}\label{fig:#1}
	\end{figure}
	\FloatBarrier
}

\newcommand{\imagenflotante}[2]{
	\begin{figure}%[!h]
		\centering
		\includegraphics[width=0.9\textwidth]{#1}
		\caption{#2}\label{fig:#1}
	\end{figure}
}



% El comando \figura nos permite insertar figuras comodamente, y utilizando
% siempre el mismo formato. Los parametros son:
% 1 -> Porcentaje del ancho de página que ocupará la figura (de 0 a 1)
% 2 --> Fichero de la imagen
% 3 --> Texto a pie de imagen
% 4 --> Etiqueta (label) para referencias
% 5 --> Opciones que queramos pasarle al \includegraphics
% 6 --> Opciones de posicionamiento a pasarle a \begin{figure}
\newcommand{\figuraConPosicion}[6]{%
  \setlength{\anchoFloat}{#1\textwidth}%
  \addtolength{\anchoFloat}{-4\fboxsep}%
  \setlength{\anchoFigura}{\anchoFloat}%
  \begin{figure}[#6]
    \begin{center}%
      \Ovalbox{%
        \begin{minipage}{\anchoFloat}%
          \begin{center}%
            \includegraphics[width=\anchoFigura,#5]{#2}%
            \caption{#3}%
            \label{#4}%
          \end{center}%
        \end{minipage}
      }%
    \end{center}%
  \end{figure}%
}

%
% Comando para incluir imágenes en formato apaisado (sin marco).
\newcommand{\figuraApaisadaSinMarco}[5]{%
  \begin{figure}%
    \begin{center}%
    \includegraphics[angle=90,height=#1\textheight,#5]{#2}%
    \caption{#3}%
    \label{#4}%
    \end{center}%
  \end{figure}%
}
% Para las tablas
\newcommand{\otoprule}{\midrule [\heavyrulewidth]}
%
% Nuevo comando para tablas pequeñas (menos de una página).
\newcommand{\tablaSmall}[5]{%
 \begin{table}
  \begin{center}
   \rowcolors {2}{gray!35}{}
   \begin{tabular}{#2}
    \toprule
    #4
    \otoprule
    #5
    \bottomrule
   \end{tabular}
   \caption{#1}
   \label{tabla:#3}
  \end{center}
 \end{table}
}

%
% Nuevo comando para tablas pequeñas (menos de una página).
\newcommand{\tablaSmallSinColores}[5]{%
 \begin{table}[H]
  \begin{center}
   \begin{tabular}{#2}
    \toprule
    #4
    \otoprule
    #5
    \bottomrule
   \end{tabular}
   \caption{#1}
   \label{tabla:#3}
  \end{center}
 \end{table}
}

\newcommand{\tablaApaisadaSmall}[5]{%
\begin{landscape}
  \begin{table}
   \begin{center}
    \rowcolors {2}{gray!35}{}
    \begin{tabular}{#2}
     \toprule
     #4
     \otoprule
     #5
     \bottomrule
    \end{tabular}
    \caption{#1}
    \label{tabla:#3}
   \end{center}
  \end{table}
\end{landscape}
}

%
% Nuevo comando para tablas grandes con cabecera y filas alternas coloreadas en gris.
\newcommand{\tabla}[6]{%
  \begin{center}
    \tablefirsthead{
      \toprule
      #5
      \otoprule
    }
    \tablehead{
      \multicolumn{#3}{l}{\small\sl continúa desde la página anterior}\\
      \toprule
      #5
      \otoprule
    }
    \tabletail{
      \hline
      \multicolumn{#3}{r}{\small\sl continúa en la página siguiente}\\
    }
    \tablelasttail{
      \hline
    }
    \bottomcaption{#1}
    \rowcolors {2}{gray!35}{}
    \begin{xtabular}{#2}
      #6
      \bottomrule
    \end{xtabular}
    \label{tabla:#4}
  \end{center}
}

%
% Nuevo comando para tablas grandes con cabecera.
\newcommand{\tablaSinColores}[6]{%
  \begin{center}
    \tablefirsthead{
      \toprule
      #5
      \otoprule
    }
    \tablehead{
      \multicolumn{#3}{l}{\small\sl continúa desde la página anterior}\\
      \toprule
      #5
      \otoprule
    }
    \tabletail{
      \hline
      \multicolumn{#3}{r}{\small\sl continúa en la página siguiente}\\
    }
    \tablelasttail{
      \hline
    }
    \bottomcaption{#1}
    \begin{xtabular}{#2}
      #6
      \bottomrule
    \end{xtabular}
    \label{tabla:#4}
  \end{center}
}

%
% Nuevo comando para tablas grandes sin cabecera.
\newcommand{\tablaSinCabecera}[5]{%
  \begin{center}
    \tablefirsthead{
      \toprule
    }
    \tablehead{
      \multicolumn{#3}{l}{\small\sl continúa desde la página anterior}\\
      \hline
    }
    \tabletail{
      \hline
      \multicolumn{#3}{r}{\small\sl continúa en la página siguiente}\\
    }
    \tablelasttail{
      \hline
    }
    \bottomcaption{#1}
  \begin{xtabular}{#2}
    #5
   \bottomrule
  \end{xtabular}
  \label{tabla:#4}
  \end{center}
}



\definecolor{cgoLight}{HTML}{EEEEEE}
\definecolor{cgoExtralight}{HTML}{FFFFFF}

%
% Nuevo comando para tablas grandes sin cabecera.
\newcommand{\tablaSinCabeceraConBandas}[5]{%
  \begin{center}
    \tablefirsthead{
      \toprule
    }
    \tablehead{
      \multicolumn{#3}{l}{\small\sl continúa desde la página anterior}\\
      \hline
    }
    \tabletail{
      \hline
      \multicolumn{#3}{r}{\small\sl continúa en la página siguiente}\\
    }
    \tablelasttail{
      \hline
    }
    \bottomcaption{#1}
    \rowcolors[]{1}{cgoExtralight}{cgoLight}

  \begin{xtabular}{#2}
    #5
   \bottomrule
  \end{xtabular}
  \label{tabla:#4}
  \end{center}
}


















\graphicspath{ {./img/} }

% Capítulos
\chapterstyle{bianchi}
\newcommand{\capitulo}[2]{
	\setcounter{chapter}{#1}
	\setcounter{section}{0}
	\chapter*{#2}
	\addcontentsline{toc}{chapter}{#2}
	\markboth{#2}{#2}
}

% Apéndices
\renewcommand{\appendixname}{Apéndice}
\renewcommand*\cftappendixname{\appendixname}

\newcommand{\apendice}[1]{
	%\renewcommand{\thechapter}{A}
	\chapter{#1}
}

\renewcommand*\cftappendixname{\appendixname\ }

% Formato de portada
\makeatletter
\usepackage{xcolor}
\newcommand{\tutor}[1]{\def\@tutor{#1}}
\newcommand{\course}[1]{\def\@course{#1}}
\definecolor{cpardoBox}{HTML}{E6E6FF}
\def\maketitle{
  \null
  \thispagestyle{empty}
  % Cabecera ----------------
\noindent\includegraphics[width=\textwidth]{cabecera}\vspace{1cm}%
  \vfill
  % Título proyecto y escudo informática ----------------
  \colorbox{cpardoBox}{%
    \begin{minipage}{.8\textwidth}
      \vspace{.5cm}\Large
      \begin{center}
      \textbf{TFG del Grado en Ingeniería Informática}\vspace{.6cm}\\
      \textbf{\LARGE\@title{}}
      \end{center}
      \vspace{.2cm}
    \end{minipage}

  }%
  \hfill\begin{minipage}{.20\textwidth}
    \includegraphics[width=\textwidth]{escudoInfor}
  \end{minipage}
  \vfill
  % Datos de alumno, curso y tutores ------------------
  \begin{center}%
  {%
    \noindent\LARGE
    Presentado por \@author{}\\ 
    en Universidad de Burgos --- \@date{}\\
    Tutor: \@tutor{}\\
  }%
  \end{center}%
  \null
  \cleardoublepage
  }
\makeatother

\newcommand{\nombre}{Francisco Saiz Güemes} %%% cambio de comando

% Datos de portada
\title{Weblectric 2018}
\author{\nombre}
\tutor{Dr. Álvar Arnaiz González \\ Dr. Jesús Maudes Raedo}
\date{\today}

\begin{document}

\maketitle


\newpage\null\thispagestyle{empty}\newpage


%%%%%%%%%%%%%%%%%%%%%%%%%%%%%%%%%%%%%%%%%%%%%%%%%%%%%%%%%%%%%%%%%%%%%%%%%%%%%%%%%%%%%%%%
\thispagestyle{empty}


\noindent\includegraphics[width=\textwidth]{cabecera}\vspace{1cm}

\noindent D. Álvar Arnaiz González y D. Jesús Maudes Raedo, profesores del departamento de Ingeniería Civil del área de Lenguajes y Sistemas Informáticos.

\noindent Exponen:

\noindent Que el alumno D. \nombre, con DNI 71567002H, ha realizado el Trabajo final de Grado en Ingeniería Informática titulado título de TFG. 

\noindent Y que dicho trabajo ha sido realizado por el alumno bajo la dirección del que suscribe, en virtud de lo cual se autoriza su presentación y defensa.

\begin{center} %\large
En Burgos, {\large \today}
\end{center}

\vfill\vfill\vfill

% Author and supervisor
\begin{minipage}{0.45\textwidth}
\begin{flushleft} %\large
Vº. Bº. del Tutor:\\[2cm]
D. nombre tutor
\end{flushleft}
\end{minipage}
\hfill
\begin{minipage}{0.45\textwidth}
\begin{flushleft} %\large
Vº. Bº. del co-tutor:\\[2cm]
D. nombre co-tutor
\end{flushleft}
\end{minipage}
\hfill

\vfill

% para casos con solo un tutor comentar lo anterior
% y descomentar lo siguiente
%Vº. Bº. del Tutor:\\[2cm]
%D. nombre tutor


\newpage\null\thispagestyle{empty}\newpage




\frontmatter

% Abstract en castellano
\renewcommand*\abstractname{Resumen}
\begin{abstract}
En la actualidad, el cliente tiene en su posesión un algoritmo de optimización al que nutre con datos para sacar una serie de resultados y posteriormente analizarlos.

En este proyecto se va a dejar apartado el proceso de análisis de esos resultados y se va a enfocar en el desarrollo de una aplicación web para facilitar al usuario final la administración de los datos previos a la ejecución de dicho algoritmo.

Se presentarán diferentes alternativas para realizar el mantenimiento de los datos así como diferentes tipos de usuarios conviviendo en la aplicación. Formularios y hojas de cálculo serán los elementos utilizados por los usuarios para realizar el mantenimiento de la información.

Estos datos se almacenarán en una base de datos MySQL a la que se realizarán peticiones para construir la simulación y generar los ficheros correspondientes para poder nutrir el algoritmo.

La aplicación web está disponible en el siguiente enlace: 

\url{https://www.proyectoubu.nesiweb.com/users/login}


\end{abstract}

\renewcommand*\abstractname{Descriptores}
\begin{abstract}
Aplicación web, base de datos MySQL, formularios, hojas de cálculo, análisis de resultados, algoritmo, simulación.
\end{abstract}

\clearpage

% Abstract en inglés
\renewcommand*\abstractname{Abstract}
\begin{abstract}
At present, the customer has in his possession an optimization algorithm that he nourishes with data in aim to obtain a series of results and then analyze them.

The results' analysis process of this project is going to be paused, and is going to be focus  in a web application development to help the user the management of the previous data to the execution of the algorithm.

Different alternatives will be presented to perform the data maintenance as well as different user types living together in the application. Forms and spreadsheets will be the elements used by the users to carry out the information maintenance.

These data will be stored in a MySQL database to which request will be made to generate the simulation and the corresponding files to feed the algorithm.

The web application is available in the following link:

\url{https://www.proyectoubu.nesiweb.com/users/login}

\end{abstract}

\renewcommand*\abstractname{Keywords}
\begin{abstract}
Web application, MySQL database, forms, spreadsheets, results' analysis, algorithm, simulation.
\end{abstract}

\clearpage

% Indices
\tableofcontents

\clearpage

\listoffigures

\clearpage

\listoftables
\clearpage

\mainmatter
\capitulo{1}{Introducción}

La optimización consiste en la selección del mejor elemento (con respecto a algún criterio) dentro de un conjunto de elementos disponibles~\cite{wiki:optimizacion}. Si hablamos de problemas de optimización, el proceso consiste en minimizar o maximizar una función real escogiendo de manera sistemática valores de entrada (tomados de un conjunto permitido) y calculando el valor de la función~\cite{wiki:optimizacion}.

Existen diferentes métodos de optimización a los que se hará referencia en algún punto de la documentación, sin embargo, el proyecto está enfocado a la administración a través de una aplicación web de unos datos que nos facilita el cliente fruto de la ejecución de un algoritmo multi-objetivo llamado \textit{Nondominated Sorting Genetic Algorithm II (NSGA-II)}~\cite{pdf:nsga-ii}.

Para la administración de todos estos datos, en la figura~\ref{fig:weblectric}, presentamos \textit{Weblectric}, una aplicación desarrollada con la idea de dar la posibilidad al usuario de ejecutar el algoritmo y visualizar los resultados.

A lo largo del proyecto podremos ver su estructura, el \textit{framework} utilizado, su diseño y algunas librerías de interés para leer y escribir datos con PHP en las hojas de cálculo. De la misma manera, reflejaremos los problemas detectados, cómo se han resuelto y qué lineas de futuro marcamos en el horizonte para continuar con el desarrollo de \textit{Weblectric}.

\begin{figure}[ht]
	\centering
	\includegraphics[width=0.4\textwidth]{/weblectric/logo}
	\caption{Weblectric.}
	\label{fig:weblectric}
\end{figure}
\capitulo{2}{Objetivos del proyecto}

Unificando y buscando complementar nuestros conocimientos y oportunidades con las necesidades manifestadas por el cliente, podríamos definir los objetivos del proyecto como:

\begin{itemize}
	\item Dar soporte a través de una aplicación web a los resultados obtenidos mediante la ejecución de un algoritmo de optimización.
	
	\item En cuanto a objetivos personales, quizá más de carácter técnico, aprender a emular en nuestra máquina, un servidor que nos permita desarrollar usando el framework CakePHP una aplicación web estable que de soporte a las necesidades del cliente.
	
	\item Conectar la aplicación con una base de datos MySql.
	
	\item Viendo que las puertas al mundo profesional se me abren por este camino, me parece interesante complementar los conocimientos obtenidos durante estos años en la universidad con otros lenguajes de programación como PHP y JavaScript.
\end{itemize}

Como objetivo final del proyecto, queremos generar el máximo valor de negocio al cliente y que la aplicación web desarrollada, cubra todas las necesidades que podamos.

\capitulo{3}{Conceptos teóricos}

Aunque el proyecto está enfocado a la administración a través de una aplicación web de unos datos que nos facilita el cliente, la base del proyecto se cimienta sobre un algoritmo de optimización capaz de proporcionarnos soluciones acerca de la mejor distribución de plantas energéticas en México. 

La solución proporcionada viene de parte de un algoritmo multiobjetivo llamado NSGA-II\footnote{Nondominated Sorting Genetic Algorithm II}.

\section{NSGA-II}

Implementando esta solución, la meta a la que se quiere llegar es a encontrar un vector de variables x=(x1,x2,…xj) que cumpla con todas las restricciones y condiciones, donde las funciones objetivos resultantes sean optimizadas. \cite{pdf:algoritmo}

Se denomina espacio de solución al conjunto de todas las combinaciones posibles. Es denotado mediante: fn(x)=z=(z1,z2,…zM). 

La diferencia entre problemas de optimizacion monoobjetivo y multiobjetivo, es que en los primeros, una solución se considera mejor que otra si con ella se obtiene una solución objetivo de menor valor si estamos minimizando, o una solución de mayor valor si estamos maximizando.

Sin embargo, en los problemas multiobjetivo, este criterio no es correcto pues entran en juego simultáneamente funciones de minimizar y de maximizar.

Para llegar a una solución, en los problemas multiobjetivo, se introduce un nuevo operador, dominancia. Que define: una solución 
x(1) domina otra solución x(2) si se cumplen las siguientes condiciones. \cite{pdf:nsga-ii}

\begin{itemize}
	\item La solución x(1) no siempre es de menor calidad que x(2) en todos los objetivos.
	\item Al menos en uno de los objetivos, la solución x(1) es estrictamente mejor que x(2).
\end{itemize}

Utilizando estas reglas de manera iterativa sobre un conjunto de soluciones de un probloema de optimización multiobjetivo, se puede llegar a establecer cuales son las alternativas dominantes. Las conocemos como Conjunto No Dominado.
El resto de soluciones pasan a formar parte del Conjunto de Soluciones Dominadas. 

Logrando establecer este conjunto de Soluciones Dominantes en un espacio objetivo, podemos hablar de Frente óptimo de Pareto. (Figura \ref{fig:frentePareto})

\begin{figure}[ht]
	\centering
	\includegraphics[width=1\textwidth]{/conceptosTeoricos/frentePareto}
	\caption{Frente de pareto en un problema de minimización.\cite{img:frente_pareto}}
	\label{fig:frentePareto}
\end{figure}

\subsection{Pseudocódigo NSGA-II \cite{pdf:nsga-ii}}

\begin{enumerate}
	\item Generar una población P de tamaño N. 
	\item Identificar los frentes de dominancia y evaluar las 
	distancias de apilamiento en cada frente. 
	\item Usando selección (<c)\footnote{Selección por torneo según operador de 
		apilamiento. La selección retorna la solución ganadora i basándose en 
		dos criterios fundamentales.  
		\begin{itemize}
			\item Si tiene mejor rango: ri<rj.
			\item Si tienen el mismo rango pero i tiene mejor distancia de apilamiento: di>d.
		\end{itemize}
	}, cruzamiento y mutación se 
	genera una población descendiente del mismo 
	tamaño de P.  	
	\item Reunir Padres e hijos en un conjunto de tamaño 2N y 
	clasificar los frentes de dominancia. 
	\item Determinar el conjunto descendiente final 
	seleccionando los frentes de mejor rango. Si se 
	supera el límite de población N, eliminar las 
	soluciones con menor distancia de apilamiento en el 
	último frente seleccionado. 
	\item Sí se cumple el criterio de convergencia, Fin del 
	proceso. De lo contrario retornar al paso 3.
\end{enumerate}
\capitulo{4}{Técnicas y herramientas}

Para el desarrollo del proyecto y para lograr alcanzar tanto los objetivos académicos como personales propuestos anteriormente, he considerado de utilidad las siguientes técnicas y herramientas.

\section{Gestión del proyecto y control de versiones}

\subsection{Gestión del proyecto}

A la hora de gestionar un proyecto, podemos clasificar el método usado para su gestión en dos grandes grupos: Metodología tradicional y metodología ágil.

Analizando las diferencias entre ellas podemos afirmar~\cite{web:diferencias-metodologias}:

\begin{itemize}
	\item En la metodología tradicional, esta presente la figura de un Project Manager que basa sus conocimientos en el \textit{Project Management Body of Knowledge}\footnote{PMBoK. Libro donde se recogen técnicas y acciones a llevar a cabo dentro de un proyecto para obtener un resultado próspero.}.
	\item La metodología tradicional se centra en un enfoque proactivo y predictivo. Busca desde los orígenes del proyecto definir todo lo definible antes de empezar, anticiparse a cualquier cambio, buscar proyección, es decir, dar un alcance lo más completo posible y ajustar el coste al máximo.
	
	\item La metodología ágil surge como necesidad del cliente a proyectos no muy grandes. No existe una necesidad por parte del cliente de una planificación inicial exhaustiva, sino que necesita un producto en un espacio corto de tiempo y no hay tiempo para grandes planificaciones.
	\item Es probable que el producto demandado por el cliente en un principio, sea diferente del demandado a final del proyecto. Esto se debe al continuo cambio sobretodo en el mundo de las TIC. El cliente sabe qué necesita pero desconoce cómo se va a concretar a $X$ días vista
\end{itemize}

Estando delante de un proyecto no muy grande y aprovechando los conocimientos recibidos en el grado, hemos decidido utilizar una metodología ágil para gestionar nuestro proyecto.

\subsubsection{SCRUM}

\textit{Scrum}\footnote{No son siglas. Su significado viene de la palabra melé. Jugada de rugby en la que jugadores de ambos equipos se agrupan en una formación en la cual lucharán por obtener el balón que se introduce por el centro. \cite{web:scrum_origen}}~\cite{web:scrum} es uno de los métodos ágiles más extendidos. Se trata de un método incremental e iterativo que divide el desarrollo del producto en ciclos llamados \textit{sprints}.

Al inicio de cada \textit{sprint}, se realiza una reunión entre todos los integrantes del proyecto donde se definen los objetivos y requisitos de cada ciclo. Cada una de esas tareas se denomina \textit{issues}.

Este tipo de metodología ha sido muy eficiente en el desarrollo. Una de las ventajas que nos ha proporcionado, ha sido la capacidad de reaccionar ante los cambios teniendo la oportunidad de ir creando en cada \textit{sprint}, especificaciones y requerimientos nuevos.

\subsubsection{ZenHub}

\textit{ZenHub} es una plataforma de gestión de proyectos que se integra en \textit{GitHub}, instalándose en el navegador mediante una extensión \footnote{Podemos descargarlo en https://www.zenhub.com/}.

Es una herramienta muy cómoda y visual, ya que permite administrar todos los elementos comentados anteriormente característicos de la metodología SCRUM. Destacar que ZenHub llama a los \textit{sprints}, \textit{milestones}. El resto de nomenclatura es igual.

En la figura \ref{fig:ZenHub} tenemos presentes las diferentes columnas del \textit{tablero}. Este elemento, es de gran utilidad para clasificar por \textit{milestones}, cada \textit{issue}.

\begin{figure}[ht]
	\centering
	\includegraphics[width=1\textwidth]{/conceptosTeoricos/ZenHub}
	\caption{Ejemplo de nuestro tablero en \textit{ZenHub}.}
	\label{fig:ZenHub}
\end{figure}

\subsection{Control de versiones}

Como repositorio y control de versiones, hemos elegido \textit{Git} a través de la herramienta \textit{GitHub}\footnote{Software de código abierto y gratuito que permite un control de versiones a través de ramas (\textit{branchs}) en las que cada usuario puede editar y publicar cambios.}

GitHub es una plataforma online basada en Git, que permite la creación de repositorios tanto públicos como privados en los que posteriormente un equipo puede alojar su trabajo.

Además, GitHub ofrece una version para escritorio (\textit{GitHub Desktop}) con la que poder realizar subidas y bajadas (\textit{push - pull}) directamente desde el escritorio. Nosotros hemos utilizado \textit{Sourcetree}, un cliente Git que proporciona una interfaz amigable para interactuar con nuestros repositorios. Pertenece a la empresa \textit{Atlassian.}

\newpage

\section{Entorno de desarrollo}

Puesto que el objetivo principal del proyecto es construir una aplicación web dando soporte al cliente a que pueda cubrir sus necesidades, necesitamos un sitio donde alojar la aplicación. 

Antes de contratar ningún servicio externo, empezamos a desarrollar la aplicación en nuestro entorno local. 

Para ello necesitamos de la siguiente herramienta.

\subsection{XAMPP}

XAMPP es un paquete de software libre que consiste principalmente en el sistema de gestión de bases de datos \textit{MySQL}, el servidor web \textit{Apache} e intérpretes para lenguajes PHP y Perl~\cite{wiki:xampp}.

La ventaja de utilizar esta herramienta es que te ahorras el tiempo y la dificultad de configuración de cada uno de los servicios por separado. 

\subsection{HeidiSQL}

Aunque XAMPP trae consigo el servicio \textit{PhpMyAdmin} para gestionar las bases de datos \textit{MySQL}, nosotros hemos decidido utilizar \textit{HeidiSQL} ya que tiene una interfaz más amigable y permite más opciones.

\textit{HeidiSQL} es un software libre de código abierto que nos da la facilidad de conectarnos a servidores \textit{MySQL}.

Para administrar las bases de datos con esta herramienta, el usuario tiene que iniciar sesión en un servidor \textit{MySQL} local o remoto. 

\subsection{Filezilla}

Una vez que tenemos nuestro entorno de desarrollo en local, es común necesitar un servidor externo donde alojar tu aplicación. Para transferir los archivos al servidor, utilizamos \textit{Filezilla}. Un gestor \textit{FTP} de código abierto y software libre.

Tras establecer la conexión con el servidor, el manejo de los archivos y la navegación ente los directorios es fácil e intuitiva. Además, permite arrastrar y soltar, lo que facilita su uso.

\subsection{Visual Studio Code}

\textit{Visual Studio Code} es un editor de código fuente desarrollado por \textit{Microsoft}. Al incluir control integrado de \textit{Git}, hace que el resultado y el control de versiones de nuestro código sea más manejable.

No es un IDE como tal, pero gracias a numerosas extensiones hace que el trato con él sea mucho más satisfactorio. Acepta extensiones para hacer más fácil la lectura de los lenguajes \textit{PHP}, \textit{HTML}, \textit{CSS} y \textit{Javascript}. 



\subsection{Prepros}

\textit{Prepros} es una herramienta completa para desarrollo front-end\footnote{La parte del software que interactúa con los usuarios.}. Se trata de un compendio de funcionalidades que abarcan desde el desarrollo con lenguajes como CSS o Javascript, a la optimización de imágenes.

Además, en este trabajo, se utiliza como compilador de archivos \textit{*.scss} a \textit{*.css} pues el desarrollo se hace en \textit{Sass}, que es parecido a \textit{CSS}. Posteriormente se compila la estructura de archivos \textit{*.scss} en un \textit{*.css} general que utiliza la web.

Para una correcta estructuración de la hoja de estilos, en la figura \ref{fig:scss} podemos ver como se ha llevado a cabo.
Tenemos un directorio con cada uno de los archivos \textit{.scss} que se agrupan en un \textit{estilos.scss} que a su vez, siendo compilado con el \textit{Prepros}, genera un \textit{estilos.css}. Este archivo es de donde se nutre la aplicación web.

\begin{figure}[ht]
	\centering
	\includegraphics[width=0.3\textwidth]{/conceptosTeoricos/scss}
	\caption{Estructura de directorios de la hoja de estilos.}
	\label{fig:scss}
\end{figure}

\newpage

\section{Frameworks }

A continuación, iremos mencionando los diferentes\textit{frameworks \footnote{Entorno de trabajo.}} que tras integrarlos entre sí, han hecho posible el desarrollo del proyecto.

\subsection{CakePHP}

\textit{CakePHP}~\cite{web:cakephp} es un \textit{framework} que facilita el desarrollo de aplicaciones web en PHP~\cite{wiki:cakephp}, utilizando el patrón de diseño MVC.

Facilita alguna ayuda para integrar \textit{Ajax}\footnote{Asynchronous JavaScript And XML (JavaScript asíncrono y XML), es una técnica de desarrollo web para crear aplicaciones interactivas. \cite{wiki:ajax}}, Javascript, formularios... por lo que hace de su uso una herramienta interesante.

Además, vienen incluidos componentes de seguridad y de sesión que para una aplicación web siempre son interesantes.

En cuanto al Modelo Vista Controlador (figura~\ref{fig:mvc}), es un patrón de arquitectura de software que separa por una parte los datos y la lógica de una aplicación y por otra, la interfaz de usuario y el módulo encargado de gestionar los eventos y las comunicaciones. 

Separar las funciones de la aplicación en modelos, vistas y controladores hace que la aplicación sea mucho más ligera.

\begin{figure}[ht]
	\centering
	\includegraphics[width=0.9\textwidth]{/conceptosTeoricos/mvc}
	\caption{Ejemplo de arquitectura MVC. Ilustración extraida de~\cite{img:MVC}}
	\label{fig:mvc}
\end{figure}

\subsection{Zurb Foundation}

\textit{Foundation}~\cite{web:foundation} es un \textit{framework} de interfaz de usuario. Proporciona una cuadrícula responsive e incluye diferentes componentes:
\begin{itemize}
	\item Interfaz de usuario HTML y CSS.
	\item Plantillas.
	\item Fragmentos de código reutilizables.
	\item Tipografías.
	\item Formularios.
	\item Botones, barras de navegación y otros componentes de interfaz usuario.
	\item Extensiones de Javascript opcionales.
\end{itemize}

\textit{Foundation} está mantenida por \href{zurb.com}{Zurb} y es un proyecto de código abierto.

La competencia más directa de \textit{Foundation} como \textit{framework de CSS}, es \textit{Bootstrap}. Aunque en nuestra aplicación hemos usado \textit{Foundation}, cabe destacar que \textit{Bootstrap} ofrece unos servicios muy parecidos y una documentación de calidad como la de \textit{Foundation}. 

Por destacar alguna ventaja a favor de \textit{Bootstrap}, podemos decir que es más popular. Tiene más plugins desarrollados y una comunidad de usuarios más grande, lo que facilita cualquier tipo de consulta en la web. Sin embargo, se ha decidido utilizar \textit{Foundation}, ya que viene con un archivo de ejemplo con el que hemos podido practicar y obtener conocimientos iniciales.

\capitulo{5}{Aspectos relevantes del desarrollo del proyecto}

\section{Tratamiento de datos}

\subsection{Lectura de datos}

Teniendo en cuenta los objetivos del proyecto ya comentados en apartados anteriores y queriendo lograr la construcción de una aplicación estable para que el usuario final pueda cargar sus datos y administrarlos, nos vimos obligados en primera instancia a buscar una librería en PHP capaz de leer una serie de datos desde un Excel, pues el cliente nos envió los datos en dicho formato, para que en un futuro fuesen administrables y poder ser alojados en nuestra base de datos.

\subsubsection{Phpspreadsheet}

Para realizar esta operación, el alumno conocía una librería llamada \textit{PHPExcel} pero que actualmente a día de hoy está deprecada, por tanto, hubo que investigar y buscar cuál era la solución actualmente. Así se consiguió dar con \textit{Phpoffice/Phpspreadsheet}. Mediante esta librería, que podemos encontrar su documentación en \href{https://phpspreadsheet.readthedocs.io/en/develop/}{phpspreadsheet}~\cite{web:spreadsheet}, se puede realizar lecturas y escrituras sobre un Excel. 

En la figura~\ref{fig:spreadsheet} se puede ver que la librería cubre con creces los requisitos necesarios, además de ofrecer alguna opción extra por si en un futuro o en próximas versiones del proyecto fuesen de interés~\cite{web:spreadsheet}.

\begin{figure}[ht]
	\centering
	\includegraphics[width=0.9\textwidth]{/aspectosRelevantes/spreadsheet}
	\caption{Formatos de ficheros soportados por Spreadsheet.}
	\label{fig:spreadsheet}
\end{figure}

Para instalar librerías en CakePHP, framework utilizado para el desarrollo del proyecto, hay diversas opciones, pero una de las más sencillas es utilizar \textit{Composer\footnote{Herramienta para la gestión de dependencias en PHP. Le permite declarar y administrar las bibliotecas de las que depende su proyecto.}}.

Para instalar \textit{Spreadsheet} mediante \textit{Composer}, desde el directorio del proyecto ejecutamos:

\begin{lstlisting}[language=bash]
composer require phpoffice/phpspreadsheet
\end{lstlisting}

Una vez la instalada, simplemente con la siguiente linea se puede importar en nuestro proyecto.

\begin{lstlisting}[language=PHP]
use PhpOffice\PhpSpreadsheet\Spreadsheet;
\end{lstlisting}

Como se ve, aunque el trato de los datos y la organización de los mismos si que ha sido un tema delicado, la instalación de las librerías en \textit{CakePHP} es bastante sencilla.

\subsection{Escritura y almacenamiento de los datos}

En cuanto al manejo de los datos, aquí si que ha habido más problemas. Se nos ha juntado la gran cantidad de datos, con que el servidor donde hemos alojado esta primera versión de la aplicación no cuenta con recursos muy elevados.

Ha habido problemas a la hora de subir al servidor archivos muy grandes pues se agotaba la memoria al procesar los datos~\footnote{Rondando las 400K registros.} y se quedaba atascado. 

En nuestro entorno de desarrollo local, el problema del tamaño de los archivos lo solucionamos modificando la propiedad $upload\_max\_file\_size$ en el archivo $php.ini$ pero al no tener acceso a este archivo en el servidor, no podemos modificarlo. 

El problema de la memoria al procesar los datos, no se ha podido solucionar ni con la sentencia $ini\_set('memory\_limit', '-1');$ para indicar que puede utilizar toda la memoria que necesite, ni con la sentencia $set\_time\_limit(0);$ para indicar que tiene el tiempo de ejecución ilimitado. 

Por esta razón es por lo que hemos tenido que fragmentar las cargas y administraciones de ciertos datos aunque lo ideal de cara a la usabilidad del usuario hubiera sido poder hacerlo todo en un mismo paso.

Para almacenar todos estos datos y poder dar al usuario la posibilidad de administrarlos, se ha construido una base de datos relacional. Para gestionarla, se ha utilizado el software \textit{HeidiSQL} que ofrece la posibilidad de conectarse a servidores \textit{MySQL}\footnote{Sistema de gestión de bases de datos relacional}.

\section{Estructura de la aplicación}

La estructura de directorios de nuestro proyecto se puede ver en la figura~\ref{fig:estructuraProyecto}. Destacamos los siguientes directorios:

\begin{itemize}
	\item \textit{config}: En la figura~\ref{fig:estructuraProyectoConfig} vemos los ficheros de configuración de nuestra aplicación. Con una importancia especial podemos destacar: 
	\begin{itemize}
		\item app.php: Archivo donde se establecen los parámetros de configuración para el email, la base de datos, los logs, la sesión, el debug, etc.
		\item constantes.php: Como indica el nombre del archivo, aquí se declaran clases con constantes comunes para poder usarlas desde el resto de la aplicación.
		\item paths.php: Definir variables globales para rutas de directorios concretos.
	\end{itemize}
		
	\item \textit{webroot}: Dentro de la estructura cliente-servidor, en la figura~\ref{fig:estructuraProyectoWebroot} se encuentra lo relacionado con el cliente. Nuestra hoja de estilos, los archivos $*.js$, las imágenes, las fuentes y el favicon.pnp\footnote{Se conoce como favicon al icono que aparece en la pestaña del navegador junto con el nombre de la aplicación.}.
	
	\item \textit{src}: Aquí se almacenarán los archivos de tu aplicación.
\end{itemize}

\begin{figure}[ht]
	\centering
	\includegraphics[width=0.3\textwidth]{/aspectosRelevantes/estructuraProyecto}
	\caption{Estructura general de un proyecto desarrollado con CakePHP.}
	\label{fig:estructuraProyecto}
\end{figure}

\begin{figure}[ht]
	\centering
	\includegraphics[width=0.3\textwidth]{/aspectosRelevantes/estructuraProyectoConfig}
	\caption{Directorio \textit{config} de nuestro proyecto.}
	\label{fig:estructuraProyectoConfig}
\end{figure}

\begin{figure}[ht]
	\centering
	\includegraphics[width=0.2\textwidth]{/aspectosRelevantes/estructuraProyectoWebroot}
	\caption{Directorio \textit{webroot} de nuestro proyecto.}
	\label{fig:estructuraProyectoWebroot}
\end{figure}

No obstante, en la documentación oficial de \textit{CakePHP}~\cite{web:estructuraCarpetasCakePHP} se explican con mayor detalle cada uno de los directorios del proyecto.

\subsection{MVC}

El utilizar \textit{CakePHP} como \textit{framework} ha facilitado mucho la estructura del proyecto pues trabaja con el patrón MVC. De esta manera, como podemos ver en la figura \ref{fig:estructuraProyectoSrc}, tenemos bien separadas las tres partes fundamentales de la aplicación. 

\begin{figure}[ht]
	\centering
	\includegraphics[width=0.2\textwidth]{/aspectosRelevantes/estructuraProyectoSrc}
	\caption{Directorio \textit{src} de nuestro proyecto.}
	\label{fig:estructuraProyectoSrc}
\end{figure}

Por un lado tenemos la conexión con nuestra base de datos en lo que serían los \textit{modelos}\footnote{Directorio \textit{Models}.}. Allí están la mayoría de consultas y operaciones contra la base de datos. 

Por otro lado tenemos la parte de los \textit{controladores}\footnote{Directorio \textit{Controller}.}. que hacen de intermediarios entre la base de datos y la vista. Aquí albergamos la parte lógica de nuestra aplicación.

Por último las \textit{vistas}\footnote{Directorio \textit{Templates}.}. Esto es lo que ve el usuario. Como ya hemos comentado, para facilitarnos el trabajo y aprovechar herramientas muy útiles ya creadas y puestas en el mercado, hemos utilizado \textit{Foundation 6.4.2}

Su implementación dentro del proyecto es muy sencilla, simplemente tenemos que colocar la siguiente linea en el \textit{layout\footnote{Vista definida como plantilla común al resto de vistas.}}:

\begin{lstlisting}[language=php]
echo $this->Html->css('lib/foundation-6.4.2/css/foundation.css');
\end{lstlisting}
\capitulo{6}{Trabajos relacionados}

En este apartado vamos a ir presentando algunos de los recursos competidores existentes en el mercado. Actualmente, ya hay varias plataformas que ofrecen un software personalizado para la optimización de energía y redes eléctricas. Iremos desglosando las diferentes opciones y destacando los detalles más relevantes.

\section{Energy Exemplar®}

\textit{Energy Exemplar®} presume de ser el líder del mercado en tecnología de simulación de energía basada en la optimización. 

Su paquete de software\footnote{Publican su primera versión en el año 2000.}, encabezado por \textit{Plexos} y \textit{Aurora}, se utiliza en todas las regiones del mundo para una amplia gama de aplicaciones, desde análisis a corto plazo hasta estudios de planificación a largo plazo. Permiten minimizar los costos operativos y de inversión, maximizar las ganancias y obtener pronósticos mucho mas precisos. Ofrecen sus servicios como el mejor software de simulación de energía eléctrica, gas y agua.  

La empresa se distribuye por todo el mundo contando con 9 oficinas repartidas por los cinco continentes.

\subsection{Plexos}

Como comentábamos antes, \textit{Plexos} es uno de los software que proporciona Energy Exemplar como solución~\cite{web:EnergyExemplarPlexos}. Combina técnicas de optimización basadas en matemática para el pronóstico con una experiencia gráfica de usuario muy potente y flexible. Presume de ofrecer lo último en gestión de datos orientados a objetos

\textit{Plexos Connect} mejora el software anterior al ofrecer computación distribuida\footnote{La computación distribuida es un modelo para resolver problemas de computación masiva utilizando un gran número de ordenadores organizados en clústeres.~\cite{web:computacionDistribuida}} a través de recursos locales y en la nube. Además ofrece ejecuciones por lotes completamente automatizadas y operaciones en tiempo real. Aloja los resultados de simulación en un repositorio central.

\subsection{Aurora}

Como afirman sus creadores~\cite{web:EnergyExemplarAurora}, \textit{Aurora} es un software de análisis y pronóstico de modelado eléctrico confiable, fácil de aprender y centrado en el usuario.

Los cambios rápidamente personalizables, sus bases de datos integradas y la interfaz fácil de usar proporcionan resultados muy satifactorios en el tiempo. Aurora permite el nivel más alto de integración de software, control de modelos y facilidad de intercambio de datos, ahorrando al usuario tiempo y dinero.

\section{Bentley Systems}

\textit{Bentley Systems} es una empresa de desarrollo de software que respalda las necesidades profesionales de los responsables de la creación y gestión de la infraestructura mundial: carreteras, aeropuertos, puentes, plantas industriales, eléctricas, etc.

Ofrece soluciones para todo el ciclo de vida del activo de infraestructura, adaptadas a las necesidades de las distintas profesiones que trabajarán con ese activo durante su ciclo de vida~\cite{web:bentleySystems}.

En la actualidad, la empresa tiene sede en: Estados Unidos, Irlanda y China.
\subsection{Advancement Academy}

Innovador programa que ofrece la posibilidad de coordinar equipos de ejecución complejos de arquitectura, ingeniería y construccion (AEC). Permite gestionar la complejidad de datos que se producen al incorporar a los contratistas.

Garantiza que sus colaboradores entiendan los procesos y productos a entregar que se esperan para una ejecución eficaz del proyecto. La herramienta ofrecerá un currículo específico para el proyecto que será escalable y flexible para adaptarse a cualquier situación~\cite{web:bentleySystemsAdvancementAcademies}.

Sus soluciones abarcan todo el ciclo de vida activo, desde el diseño hasta la construcción, pasando por la optimización de las centrales.

En el siguiente \href{https://www.bentley.com/es/project-profiles}{\textit{enlace}}, puede conocer algunos de los proyectos llevados a cabo por \textit{Bentley Systems}.

\section{LEAP}

\textit{LEAP}, el sistema de planificación de alternativas energéticas a largo plazo, es una herramienta de software adoptada y utilizada por 190 países en todo el mundo para el análisis de políticas energéticas y la evaluación del ahorro energético desarrollada en el Instituto de Medio Ambiente de Estocolmo~\cite{web:LEAP}.

Entre sus usuarios, tenemos presentes a académicos, agencias gubernamentales, organizaciones no gubernamentales, empresas consultoras y empresas de energía. Su uso varía desde ciudades y estados hasta aplicaciones nacionales, regionales y globales.

En los últimos tiempos, se está convirtiendo en el estándar de facto para los países que realizan una planificación integrada de recursos, evaluaciones de mitigación de gases de efecto invernadero (GEI) y Estrategias de Desarrollo de Baja Emisión (LEDS).

Al menos 32 países utilizaron el \textit{LEAP} para crear escenarios de energía y emisiones.

Es una herramienta de modelado integrada y basada en escenarios cuyo uso puede ser rastrear la producción, el consumo de energía y la extracción de recursos en todos los campos de una economía. 

De cara al usuario, cuenta con buena reputación porque ha sido capaz de hacer transparente el uso de conceptos complejos de análisis de energía. Del mismo modo, es flexible para usuarios que si cuentan con conocimientos a fondo y experiencia.

No es un modelo de un sistema de energía en particular. Es una herramienta que se puede utilizar para cubrir unas necesidades al crear modelos de diferentes sistemas de energía, donde cada modelo requiere estructuras de datos personales y únicas. Es compatible con una amplia gama de diferentes metodologías de modelado a medio y largo plazo~\cite{web:LEAP}.

Una gran parte de los estudios realizados, utilizan un período de pronóstico de entre 20 y 50 años. Sin embargo, aunque suele ser lo habitual, no siempre es así. Por ejemplo, para los cálculos del sector eléctrico, el año se suele dividir en diferentes ``intervalos de tiempo'' definidos por el usuario para representar periodos de tiempo como pueden ser temporadas, días o incluso horas con un valor especial en el día.

La aplicación se distribuye en diferentes vistas/pantallas que hace que el usuario pueda interaccionar con el software.

\begin{itemize}
	\item Vista de análisis.
	\item Herramientas para crear modelos.
	\item Informe de resultados.
	\item Balances de energía.
	\item Diagramas de Sankey.\footnote{Se utilizan para visualizar los flujos de balance de energía para cualquier área que se esté modelando en LEAP.}
	\item Base de datos de tecnología y medio ambiente (TED).
	\item Notas y documentación.
\end{itemize}

\section{OSeMOSYS}

Alternativa de código abierto para la evaluación integrada y la planificación energética a largo plazo. El proyecto nace en 2008 durante una presentación en París.

Diseñado para aquellos que no pueden o deseen hacer una inversión financiera inicial. Cuenta con una curva de aprendizaje rápida y un compromiso de operación de poco tiempo. Gracias a su transparencia, su uso se puede encontrar como herramienta de difusión y captación.

\textit{Model Management Infrastructure (MoManI)} es una interfaz de código abierto basada en navegador enfocada en el modelado de sistemas energéticos. La estructura de la aplicación ayuda a disminuir la complejidad perceptible de \textit{OSeMOSYS} y su estructura permite que varios equipos colabores simultáneamente desde cualquier parte del mundo.
El usuario puede actualizar y editar sin dificultad cualquier parte del proceso de modelado: desde la visualización de los resultados hasta las ecuaciones matemáticas subyacentes de \textit{OSeMOSYS}.

\section{Artículos científicos}

El 15 de Junio de 2016, Khizir Mahmud,Graham E. Town publicó un artículo~\cite{pdf:articuloAplicacionesRelacionadas} en el que hacía una revisión de las herramientas informáticas presentes en el mercado que sirviesen para modelar los requisitos energéticos de los vehículos eléctricos y su imparto en las redes de distribución de la energía. El artículo tuvo un gran impacto y los proyectos comentados anteriormente hacen referencia al artículo citado. 



\capitulo{7}{Conclusiones y Líneas de trabajo futuras}

En esta sección se exponen las conclusiones obtenidas así como comentarios de interés útiles para poder seguir con el desarrollo del proyecto.

\section{Conclusiones}

\section{Líneas de trabajo futuras}





\bibliographystyle{plain}
\bibliography{bibliografia}

\end{document}
