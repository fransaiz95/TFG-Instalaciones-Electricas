\capitulo{1}{Introducción}

La optimización consiste en la selección del mejor elemento (con respecto a algún criterio) dentro de un conjunto de elementos disponibles~\cite{wiki:optimizacion}. Si hablamos de problemas de optimización, el proceso consiste en minimizar o maximizar una función real escogiendo de manera sistemática valores de entrada (tomados de un conjunto permitido) y calculando el valor de la función~\cite{wiki:optimizacion}.

Existen diferentes métodos de optimización a los que se hará referencia en algún punto de la documentación, sin embargo, el proyecto está enfocado a la administración a través de una aplicación web de unos datos que nos facilita el cliente fruto de la ejecución de un algoritmo multi-objetivo llamado \textit{Nondominated Sorting Genetic Algorithm II (NSGA-II)}~\cite{pdf:nsga-ii}.

Para la administración de todos estos datos, en la figura~\ref{fig:weblectric}, presentamos \textit{Weblectric}, una aplicación desarrollada con la idea de dar la posibilidad al usuario de ejecutar el algoritmo y visualizar los resultados.

A lo largo del proyecto podremos ver su estructura, el \textit{framework} utilizado, su diseño y algunas librerías de interés para leer y escribir datos con PHP en las hojas de cálculo. De la misma manera, reflejaremos los problemas detectados, cómo se han resuelto y qué lineas de futuro marcamos en el horizonte para continuar con el desarrollo de \textit{Weblectric}.

\begin{figure}[ht]
	\centering
	\includegraphics[width=0.4\textwidth]{/weblectric/logo}
	\caption{Weblectric.}
	\label{fig:weblectric}
\end{figure}