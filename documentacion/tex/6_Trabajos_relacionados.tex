\capitulo{6}{Trabajos relacionados}

En este apartado vamos a ir presentando algunos de los recursos competidores existentes en el mercado. Actualmente, ya hay varias plataformas que ofrecen un software personalizado para la optimización de energía y redes eléctricas. Iremos desglosando las diferentes opciones y destacando los detalles más relevantes.

\section{Energy Exemplar®}

\textit{Energy Exemplar®} presume de ser el líder del mercado en tecnología de simulación de energía basada en la optimización. 

Su paquete de software\footnote{Publican su primera versión en el año 2000.}, encabezado por \textit{Plexos} y \textit{Aurora}, se utiliza en todas las regiones del mundo para una amplia gama de aplicaciones, desde análisis a corto plazo hasta estudios de planificación a largo plazo. Permiten minimizar los costos operativos y de inversión, maximizar las ganancias y obtener pronósticos mucho mas precisos. Ofrecen sus servicios como el mejor software de simulación de energía eléctrica, gas y agua.  

La empresa se distribuye por todo el mundo contando con 9 oficinas repartidas por los cinco continentes.

\subsection{Plexos}

Como comentábamos antes, \textit{Plexos} es uno de los software que proporcione Energy Exemplar como solución~\cite{web:EnergyExemplarPlexos}. Combina técnicas de optimización basadas en matemática para el pronóstico con una experiencia gráfica de usuario muy potente y flexible. Presume de ofrecer lo último en gestión de datos orientados a objetos

\textit{Plexos Connect} mejora el software anterior al ofrecer computación distribuida\footnote{La computación distribuida es un modelo para resolver problemas de computación masiva utilizando un gran número de ordenadores organizados en clústeres.~\cite{web:computacionDistribuida}} a través de recursos locales y en la nube. Además ofrece ejecuciones por lotes completamente automatizadas y operaciones en tiempo real. Aloja los resultados de simulación en un repositorio central.

\subsection{Aurora}

Como afirman sus creadores~\cite{web:EnergyExemplarAurora}, \textit{Aurora} es un software de análisis y pronóstico de modelado eléctrico confiable, fácil de aprender y centrado en el usuario.

Los cambios rápidamente personalizables, sus bases de datos integradas y la interfaz fácil de usar proporcionan resultados muy satifactorios en el tiempo. Aurora permite el nivel más alto de integración de software, control de modelos y facilidad de intercambio de datos, ahorrando al usuario tiempo y dinero.

\section{Bentley Systems}

\textit{Bentley Systems} es una empresa de desarrollo de software que respalda las necesidades profesionales de los responsables de la creación y gestión de la infraestructura mundial: carreteras, aeropuertos, puentes, plantas industriales, eléctricas, etc.

Ofrece soluciones para todo el ciclo de vida del activo de infraestructura, adaptadas a las necesidades de las distintas profesiones que trabajarán con ese activo durante su ciclo de vida~\cite{web:bentleySystems}.

En la actualidad, la empresa tiene sede en: Estados Unidos, Irlanda y China.
\subsection{Advancement Academy}

Innovador programa que ofrece la posibilidad de coordinar equipos de ejecución complejos de arquitectura, ingeniería y construccion (AEC). Permite gestionar la complejidad de datos que se producen al incorporar a los contratistas.

Garantiza que sus colaboradores entiendan los procesos y productos a entregar que se esperan para una ejecución eficaz del proyecto. La herramienta ofrecerá un currículo específico para el proyecto que será escalable y flexible para adaptarse a cualquier situación~\cite{web:bentleySystemsAdvancementAcademies}.

Sus soluciones abarcan todo el ciclo de vida activo, desde el diseño hasta la construcción, pasando por la optimización de las centrales.

En el siguiente \href{https://www.bentley.com/es/project-profiles}{\textit{enlace}}, puede conocer algunos de los proyectos llevados a cabo por \textit{Bentley Systems}.

\section{LEAP}

\textit{LEAP}, el sistema de planificación de alternativas energéticas a largo plazo, es una herramienta de software adoptada y utilizada por 190 países en todo el mundo para el análisis de políticas energéticas y la evaluación del ahorro energético desarrollada en el Instituto de Medio Ambiente de Estocolmo~\cite{web:LEAP}.

Entre sus usuarios, tenemos presentes a académicos, agencias gubernamentales, organizaciones no gubernamentales, empresas consultoras y empresas de energía. Su uso varía desde ciudades y estados hasta aplicaciones nacionales, regionales y globales.

En los últimos tiempos, se está convirtiendo en el estándar de facto para los países que realizan una planificación integrada de recursos, evaluaciones de mitigación de gases de efecto invernadero (GEI) y Estrategias de Desarrollo de Baja Emisión (LEDS).

Al menos 32 países utilizaron el \textit{LEAP} para crear escenarios de energía y emisiones.

Es una herramienta de modelado integrada y basada en escenarios cuyo uso puede ser rastrear la producción, el consumo de energía y la extracción de recursos en todos los campos de una economía. 

De cara al usuario, cuenta con buena reputación porque ha sido capaz de hacer transparente el uso de conceptos complejos de análisis de energía. Del mismo modo, es flexible para usuarios que si cuentan con conocimientos a fondo y experiencia.

No es un modelo de un sistema de energía en particular. Es una herramienta que se puede utilizar para cubrir unas necesidades al crear modelos de diferentes sistemas de energía, donde cada modelo requiere estructuras de datos personales y únicas. Es compatible con una amplia gama de diferentes metodologías de modelado a medio y largo plazo~\cite{web:LEAP}.

Una gran parte de los estudios realizados, utilizan un período de pronóstico de entre 20 y 50 años. Sin embargo, aunque suele ser lo habitual, no siempre es así. Por ejemplo, para los cálculos del sector eléctrico, el año se suele dividir en diferentes "intervalos de tiempo" definidos por el usuario para representar periodos de tiempo como pueden ser temporadas, días o incluso horas con un valor especial en el día.

La aplicación se distribuye en diferentes vistas/pantallas que hace que el usuario pueda interaccionar con el software.

\begin{itemize}
	\item Vista de análisis.
	\item Herramientas para crear modelos.
	\item Informe de resultados.
	\item Balances de energía.
	\item Diagramas de Sankey.\footnote{Se utilizan para visualizar los flujos de balance de energía para cualquier área que se esté modelando en LEAP.}
	\item Base de datos de tecnología y medio ambiente (TED).
	\item Notas y documentación.
\end{itemize}

\section{OSeMOSYS}

Alternativa de código abierto para la evaluación integrada y la planificación energética a largo plazo. El proyecto nace en 2008 durante una presentación en París.

Diseñado para aquellos que no pueden o deseen hacer una inversión financiera inicial. Cuenta con una curva de aprendizaje rápida y un compromiso de operación de poco tiempo. Gracias a su transparencia, su uso se puede encontrar como herramienta de difusión y captación.

\textit{Model Management Infrastructure (MoManI)} es una interfaz de código abierto basada en navegador enfocada en el modelado de sistemas energéticos. La estructura de la aplicación ayuda a disminuir la complejidad perceptible de \textit{OSeMOSYS} y su estructura permite que varios equipos colabores simultáneamente desde cualquier parte del mundo.
El usuario puede actualizar y editar sin dificultad cualquier parte del proceso de modelado: desde la visualización de los resultados hasta las ecuaciones matemáticas subyacentes de \textit{OSeMOSYS}.

\section{Artículos científicos}

El 15 de Junio de 2016, Khizir Mahmud,Graham E. Town publicó un artículo~\cite{pdf:articuloAplicacionesRelacionadas} en el que hacía una revisión de las herramientas informáticas presentes en el mercado que sirviesen para modelar los requisitos energéticos de los vehículos eléctricos y su imparto en las redes de distribución de la energía. El artículo tuvo un gran impacto y los proyectos comentados anteriormente hacen referencia al artículo citado. 


